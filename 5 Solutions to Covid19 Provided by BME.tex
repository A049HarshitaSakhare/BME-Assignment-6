\documentclass[12pt]{article}

\usepackage{graphicx}
\graphicspath{{Images/}}

\usepackage{hyperref}
\hypersetup{colorlinks=true,citecolor=black,linkcolor=black,urlcolor=black}






\begin{document}
\title{\huge National Institute of Technology Raipur}
\begin{figure}
\centering
\includegraphics[scale=0.2]{NITRR.jpg}
\end{figure}
\author{\textit{Submitted By:- Harshita Upendra Sakhare}\\ \textbf{Roll.No:- 21111049}\\ \textit{Submitted To:- Prof.Saurabh Gupta}\\ \textsc{Assignment-6}\\ \textsc{ON}\\ \textbf{5 Solutions to Covid19 provided by Biomedical Engineers}}

\maketitle
\clearpage
\tableofcontents
\clearpage

\section{Acknowledgement}
\hspace{1cm}
In successful completion of my assignment on 5 SOLUTIONS TO COVID19 PROVIDED BY BIOMEDICAL ENGINEERS, I would like to thank my
Professor. Saurabh Gupta Lecturer of Biomedical Engineering, who
has guided and assisted me to complete the assignment. Without
his support I would not have finished the assignment within time.
I would also like to take this opportunity to thank my friends
and family members,\\without them this assignment could not have
been completed in a short duration.
\clearpage
\section{Introduction}
\hspace{1cm}
Biomedical engineering (BME) or medical engineering is the application of engineering principles and design concepts to medicine and biology for healthcare purposes (e.g., diagnostic or therapeutic). BME is also traditionally known as "bioengineering", but this term has come to also refer to biological engineering. This field seeks to close the gap between engineering and medicine, combining the design and problem-solving skills of engineering with medical biological sciences to advance health care treatment, including diagnosis, monitoring, and therapy.[1][2] Also included under the scope of a biomedical engineer is the management of current medical equipment in hospitals while adhering to relevant industry standards. This involves making equipment recommendations, procurement, routine testing, and preventive maintenance, a role also known as a Biomedical Equipment Technician (BMET) or as clinical engineering.
\section{Solutions Provide By Biomedical Engineers}
\hspace{1cm}
Biomedical Engineering Tools for Management of Patients with COVID-19 presents biomedical engineering tools under research (and in development) that can be used for the management of COVID-19 patients, along with BME tools in the global environment that curtail and prevent the spread of the virus. BME tools covered in the book include new disinfectants and sterilization equipment, testing devices for rapid and accurate COVID-19 diagnosis, Internet of Things applications in COVID-19 hospitals, analytics, Data Science and statistical modeling applied to COVID-19 tracking, Smart City instruments and applications, and more. Later sections discuss smart tools in telemedicine and e-health. Biomedical engineering tools can provide engineers, computer scientists, clinicians and other policymakers with solutions for managing patient treatment, applying data analysis techniques, and applying tools to help the general population curtail spread of the virus.

\clearpage

\section{Solutions For COVID-19 By Engineers}
\hspace{1cm}

\textbf{Oxygen}

\hspace{2cm}


The first form for mild respiratory insufficiency is usually the supply of extra oxygen through a nasal cannula or a more intrusive face mask. Most of the time, the oxygen comes in the form of cylinders, either small for portability or large for stationary patients and longer-term supply.

Oxygen concentrators represent an attractive alternative to tanks although this is not typically in use while caring for COVID-19 patients in hospital settings. Oxygen concentrators extract oxygen from the air on demand and supply it directly to the patient. Concentrators come in a variety of sizes from a portable shoulder bag form factor, to higher capacity stationary machines for patients who need oxygen 24/7.


\hspace{3cm}

\textbf{Continuous Positive Airway Pressure (CPAP)}

\hspace{2cm}


The next step up in treating COVID-19 patients is Continuous Positive Airway Pressure (CPAP) which is initially intended to prevent airways collapse in sleep apnoea patients, but has been shown to be beneficial to COVID patients if applied early enough in the progression of the disease.

A well-fitted face mask is an essential component of a CPAP system and as such it is quite intrusive. CPAP is only appropriate for patients who are capable of some breathing strength as CPAP effectively opposes some resistance to expiration. Variants exist that either automatically adjust the level of pressure to the patients breathing characteristics (APAP) or have different levels of pressure for inspiration and expiration (BiPAP). CPAP usually supplies (filtered) air to the patient but most masks have a port for supplementing the supply with oxygen.



\hspace{3cm}


\textbf{Ventilators}

\hspace{2cm}


Patients who cannot breathe spontaneously need to be put on a ventilator. Ventilators are capable of replacing the breath function and patients in an advanced state of respiratory distress are usually intubated and sedated at the beginning of the treatment.

Ventilators are capable of replacing the breath function and patients in an advanced state of respiratory distress are usually intubated and sedated at the beginning of the treatment. They are complex systems providing the healthcare professionals with a lot of flexibility to adapt the assisted breathing settings and to be able to wean recovering patients off the ventilator gradually.

Modern ventilators are typically closed loop pressure controlled and capable of detecting spontaneous breathing to synchronise assistance for recovering patients. They also enable the control of the composition of the gas the patient breathes from normal air to 100 percent oxygen, usually taking their supply from the hospital’s gas supply network but can also be coupled to oxygen tanks or oxygen concentrators if used in a setting where there is no gas network.


\hspace{3cm}


\textbf{Patient Monitoring}


\hspace{2cm}


An essential element of the ICU equipment is the monitoring equipment that keeps track of some of the patient vitals especially when they are ventilated and sedated but also during their recovery phase to ensure the regime of ventilation is optimised for their condition. Ventilators already provide their set of patient parameters, but usually patient monitors are separate devices as they continue to be useful after the patient can resume breathing on their own unassisted.

One of the key parameters for COVID-19 patient is the amount of oxygen in their bloodstream (SpO2), measured by pulse oximetry which uses optics within a finger clamp. Pulse oximetry tends to be used for the duration of the patient’s stay in ICU.


\hspace{3cm}


\textbf{Innovating In a Pandemic}


\hspace{2cm}


Beyond the Ventilator Challenge, the pandemic inspired engineers around the country to many innovations. This section list only a few of the innovations the authors are aware if and doesn’t mean to single them out from all the great work which is taking place.

Bioengineers at DNA Nudge developed the COVID Nudge< test from scratch during the pandemic.


\clearpage


\textbf{Personal Protective Equipment}


\hspace{2cm}


The COVID-19 pandemic has evidenced the fragility of society and the need for effective and practical ways to protect it. For the general public, the use of face masks as personal protection equipment (PPE) remain the most practical line of defence against SARS-CoV-2 as well as other respiratory viral infections.


\section{Conclusion}
\hspace{1cm}
The fight against COVID-19 has helped highlight the work and contributions of so many professionals in the bioengineering fields who are working tirelessly to help our health services cope. Their innovation and ingenuity are paving the way to successfully beat this virus. We must continue to support these fields as we evolve our health systems to deal with the challenges of healthcare in the future.
\section{Reference}
\begin{itemize}
\item Wikipidea
\item worldscientific.com
\item imeche.org
\end{itemize}




\end{document}